\section{Summary}

The thesis is part of the health app project, that wants to encourage people to have a healthier and more active lifestyle, and aims to achieve this goal through the implementation of a mobile application.
\vspace{2ex}

\noindent The project's objective is strongly motivated by the increasingly sedentary trend. In fact, many relevant organizations are noticing this negative trend and encouraging people to be more active in order to avoid several health problems, like heart and respiratory problems, but also other diseases that are less likely to happen but still more frequent due to a sedentary lifestyle. Given the lack of physical activity and the increasingly widespread adoption of technological tools, the intention is to use technology for health benefits, exploiting smartphones, which are increasingly present in everyday life, but also less popular devices like wearables, that are increasingly taking hold. The latter ones in particular help a lot in collecting metrics related to the user health status. Those wearable devices would help create a clearer picture of the user and, coupled with the mobile application, to give him a more complete experience. 
\vspace{2ex}

\noindent The application allows its users to view through charts data related to metrics like physical activity, sleep and nutrition. Most wearables are supported, and the application retrieves data previously collected from them and then performs data visualization. Moreover, a section is available where the user can easily view and edit his personal information entered when they registered for the first time (such as username, age, height, gender) and his activity goals. The possibility of provide informations by inserting them is also made easily accessible. An integrated lessons and quizzes system, combined with periodic notifications that prompt the user into entering information, allows to keep them actively involved. Also, the entire application has multi-language support, allowing the user to feel more comfortable in using it and improving its intuitiveness. In addition, a web application was also conceived to assist the admin into editing some application parameters and to help him into lesson and quizzes management. 
\newpage

\noindent For the development of the application, the flutter framework was used, which thanks to his strong support from Google and an easy learning curve, made the development easier. The language used on which Flutter is based is Dart, which has enabled the creation of a performant and responsive application. Additionally, Google Firebase was used for backend services, chose since provides real-time database, authentication, cloud storage and performance evaluation tools in an unique solutions, and also for his easy flutter integration. Also health platforms were used on the application side, depending on the specific operating system, to collect health and fitness data. All the development was performed by using the Android Studio IDE, which has a focus on mobile development, particularly on the Android side, but also on Flutter with the dedicated plugin. Lots of functionalities offered by this IDE (intelligent code editor, android emulator, APK analyzer) allowed to streamline the development process, making it more efficient. Regarding the web application instead, the React framework and the Visual Studio Code IDE were used.
\vspace{0.5ex}

\noindent The thesis activity was mainly structured into four phases: In the first one, health and well being topic has been examined, in order to understand the main guidelines to consider and the best practices to follow. In the second phase the overall system was designed, firstly by defining the requirements, and then by considering the system architecture in line with the previously defined requirements. The third phase covered the whole technological stack employed for the project. The fourth phase involved the actual implementation of the whole system architecture, which involved the applications as well as the backend part. Finally, the fifth phase regarded the final testing of the application system, by examining the achieved performances and the possible future improvements that could be made.