\section{Abstract}

The spread of smartphones and wearable devices across the years have contributed to the development of new technologies. Wearable devices, thanks to the possibility of monitoring the user and collecting data, have made possible technology integration especially in the health sphere. However, despite technological developments in this area, sedentary lifestyle remains a common and widespread problem. The Health App project, which is a collaboration between the Azienda Ospedaliera di Verona, University of Sydney and the polito DAUIN (Dipartimento di Automatica e Informatica), aims to further implement and adapt the alredy partially developed application in order to address this challenge. The goal is to enrich and adapt the existing application by providing an application which encourages the user to be more active with daily goal to meet, but also by prompting him to insert information periodically, as well as to provide for him recommendation in order to improve his actual lifestyle and achieve the best health condition possible. All this was done by proving intuitive charts to visualize the data, a user friendly way to insert data, as well as a quiz system to further engage the users, all coupled with a LLM model capable to give to the user a personalized experience.

\noindent Health guidelines set by leading organizations in the field were considered as the first thing in framing the application, moving then into how technology can help achieve an healthier lifestyle. By having that in mind, the updated requirements related to the application (both functional and non-functional) were added and refined where needed. From that point, also the architecture was consequently redefined, while the framework and the technologies employed for the development were kept for the most part, allowing to keep the already existing codebase. Another important factor was that the framework (Flutter) is pretty easy to learn and is strongly supported by Google that created it. After that it follows the actual implementation, with the new contributions made on the application, and a particular focus on the candidate contributions. The application performance that have been obtained were considered, by testing on different devices with different hardware specifications. Finally, the results that have been achieved are discussed, as well as possible future enhancements, by highlighting interesting features that could be added to the application. 
