\section{Abstract}

The spread of smartphones and wearable devices across the years have contributed to the development of new technologies. Wearable devices, thanks to the possibility of monitoring the user and collecting data, have made possible technology integration especially in the health sphere. However, sedentary lifestyle remains a common and widespread problem. The health app project aims to address this challenge through the implementation of a mobile application, called Well Being App. The goal is to develop an application capable of providing functionalities which encourages the user to be more active with daily goal to meet, but also by prompting him to insert information periodically, in order to improve his lifestyle and achieve the best health condition possible. All this was done by providing intuitive charts to visualize the data, a user friendly way to insert data, as well as a quiz and a notification system to engage the user. Furthermore, sign-up information together with activity goals are easily accessible and editable, and wearable integration has been made, all coupled with a multi language support, to provide a better user experience and make the application more intuitive and easy to use.

\noindent Health guidelines set by leading organizations in the field were firstly considered in framing the application, moving then into how technology can help achieve an healthier lifestyle. By having that in mind, the requirements (both functional and non-functional) were defined. From that point, also the architecture was consequently defined, and the technology stack chosen. Flutter covered the frontend part while Google Firebase the backend, chosen since it provides a comprehensive solution for real-time database, authentication, and cloud storage, as well as an easy integration with flutter. After that follows the implementation of the system architecture, with the contributions made on the application, particularly on the backend part. Finally, the application system performance were assessed and the achieved results were discussed, as well as possible future enhancements and potential new features. 

\subsection{Key Words:}

Health, Sedentary Lifestyle, Wearable Devices, Google Firebase SDK, Mobile Application, Flutter, User Engagement, System Architecture, System Performance.