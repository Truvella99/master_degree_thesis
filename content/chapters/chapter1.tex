\setcounter{section}{0}
\section{Health and Well Being}
Health is one of the most important, if not the most important aspects of a person's life. For this reason, over the years, different organisations have established different guidelines on how to stay healthy, thus increasing people's life expectancy and quality of life. Among these, the most widely worldwide recognized is the World Health Organization (WHO) \cite{Who} that provides several guidelines not only in term of physical activity, covering also other health aspects. \newline As far as concerns the physical activity, the WHO estimates that 1 in 3 adults and 4 in 5 adolescents do not do enough physical activity, with adolescents girls less active than adolescents boys and with inactivity levels that increases after 60 years of age. This level is expected to rise due to country economic development (more use of technology, change of cultural values and more sedentary behaviour). This trend sadly keep going in the wrong direction, despite the fact that physical activity has countless benefits, like reducing the risk of heart disease, cancer, diabetes, hypertension and depression.
\begin{table}
    \begin{itemize}[nosep] % 'nosep' removes extra spacing between items
        \item Children and Adolescents:\vspace{2ex}
              \begin{itemize}[nosep]
                  \item \textbf{Regular physical activity} enhances fitness, cardiometabolic health, bone strength, cognitive and mental health while reducing body fat.
                  \item \textbf{Sedentary behavior} leads to increased adiposity, poorer cardiometabolic health, behavioral issues, and reduced sleep duration.
              \end{itemize}
              \vspace{3ex}
        \item Adults and Older Adults:\vspace{2ex}
              \begin{itemize}[nosep]
                  \item \textbf{Active adults} experience lower body fat, risks of all-cause mortality, cardiovascular diseases, hypertension, specific cancers, and type-2 diabetes. They also enjoy improved mental health, cognitive function, and sleep quality.
                  \item \textbf{Sedentary lifestyles} are associated with higher mortality rates and increased incidences of chronic diseases like cardiovascular issues and cancer.
              \end{itemize}
              \vspace{3ex}
        \item Pregnant and Post-Partum Women:\vspace{2ex}
              \begin{itemize}[nosep]
                  \item \textbf{Engaging in physical activity} decreases the risks of pre-eclampsia, gestational hypertension, gestational diabetes, excessive weight gain, newborn complications, and postpartum depression, while having no negative effects on birth weight or stillbirth risk.
                        %\item Sedentary behavior can lead to complications for both the mother and newborn, emphasizing the importance of maintaining activity during and after pregnancy.
              \end{itemize}
    \end{itemize}
    \caption*{Active vs Sedentary lifestyle\cite{WhoPhysicalActivityBenefits}.}
\end{table}
\newline Food is also crucial in order to be healtier. Having a healthy diet helps to prevent several diseases (like heart disease, diabetes and cancer) and also malnutrition in all its forms. However, care has to be taken in choosing the right food sources that have good quality and avoid processed foods. Eating noble foods like fruits, vegetables, legumes, nuts, and whole grains, while limiting the intake of salt, sugar, and fats, is the key to a healthy diet. For all these reasons, both physical activity and diet are strongly promoted by the WHO through his global action plan, by calling international partners, private sector and also civil society to take action in order to support them. 
\subsection{Guidelines}
\subsubsection{Physical Activity Guidelines}
As far as concerns the physical activity, the WHO gives some recommendation based on the age group \cite{WhoPhysicalActivityGuidelines}:
\vspace{3ex}
\begin{itemize}[nosep] % 'nosep' removes extra spacing between items
    \item 5-17 years:\vspace{2ex}
          \begin{itemize}[nosep]
              \item Should do at least 60 minutes of physical activity with moderate/vigorous-intensity daily (of course more than 60 minutes provides additional benefits), as well as bone-strengthening and muscle-strengthening activities.
          \end{itemize}
          \vspace{3ex}
    \item 18-64 years:\vspace{2ex}
          \begin{itemize}[nosep]
              \item Should do at least 150 minutes of physical activity with moderate-intensity in a week or at least 75 minutes of physical activity with vigorous-intensity in a week or an equivalent combination of both (increasing moderate-intensity will provide additional benefits), but also muscle-strengthening activities by involving major muscle groups.
          \end{itemize}
          \vspace{3ex}
    \item 65 years and above:\vspace{2ex}
          \begin{itemize}[nosep]
              \item Should do at least 150 minutes of physical activity with moderate-intensity in a week or at least 75 minutes of physical activity with vigorous-intensity in a week or an equivalent combination of both (increasing moderate-intensity will provide additional benefits), recruiting major muscle groups with muscle-strengthening activities but also including exercises to enhance balance and prevent falls in case of poor mobility.
          \end{itemize}
\end{itemize}
\subsubsection{Healthy Diet Guidelines}
Regarding having an healthy diet, also here the WHO gives some guidelines, emphasizing that a good diet includes legumes, fruit, vegetables, animal sources foods (like meat, fish, eggs, and milk), cereals (like wheat and barley) and also tubers (like potato and yam). It also gives some further recommendations\cite{WhoHealthyDietGuidelines}: 
\vspace{3ex}
\begin{itemize}[nosep] % 'nosep' removes extra spacing between items
    \item Babies and young children breastfeeding:\vspace{2ex}
          \begin{itemize}[nosep]
              \item Breastfeeding promotes healthy growth, as well as having long-term benefit, like reducing the risk of developing nonncommunicable diseases, overweight, obesity. From birth until 6 month of like is important to feed the baby only with breastmilk, while from 6 month to 2 years of age is important to introduce also additional complementary foods, while still breastfeeding.
          \end{itemize}
          \vspace{3ex}
    \item Eat lots of vegetables and fruit:\vspace{2ex}
          \begin{itemize}[nosep]
              \item These foods are rich in vitamins, minerals, dietary fiber, antioxidants and plant protein, which help to prevent heart disease, stroke, diabetes, obesity and some cancers.
          \end{itemize}
          \vspace{3ex}
    \item Eat less fat:\vspace{2ex}
          \begin{itemize}[nosep]
              \item Fats and oils are concentrated source of energy, so it is important to limit them, especially saturated and industrially-produced trans-fat that can increase the risk of stroke and heart disease. To avoid gaining weight in an unhealthy way because of them, care has to be taken in using unsaturated vegetable oils (like olive oil) instead of animal fats or oils high in saturated fats (like butter or palm) and in any case fat consumption should not exceed 30\% of total energy intake.
          \end{itemize}
          \vspace{3ex}
    \item Limit sugars:\vspace{2ex}
          \begin{itemize}[nosep]
              \item Sugar consumption should be the 10\% of total energy intake. This should be achieved by limiting soft drinks, soda and other drinks high in sugars (fruit juices or yogurt drinks) and also by avoiding the consumption of processed foods high in sugars (like cookies, cakes, chocolate). Better to choose fresh fruits instead of them.
          \end{itemize}
\end{itemize}
\vspace{3ex}
The WHO also suggests to limit the intake of free sugars, salt, and saturated fats, as well as to avoid the consumption of processed foods.
\subsection{Technology Role in Health}
\subsubsection{Smartphone Application}
\subsubsection{Wearable Devices}