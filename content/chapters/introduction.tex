\section*{Introduction}
\addcontentsline{toc}{section}{Introduction}
The work that is being presented in the next pages is based on two main arguments: the importance of following a healthy lifestyle and how technology can help in achieving this goal. Infact, our project work along with my colleagues concerned the further implementation of a partially developed application suited for achieving a good lifestyle. The application, composed both by a frontend part and a backend one, allow the users to track their data regarding metrics like nutrition, as well as physical activity, sleep and emotional state. These data can be either inserted manually in an easy way or collected through a wearable device, and are intuitively displayed through charts. The application still requires the user to insert his basics information (username, age, sex, weight, height), which can be modified later in a dedicated section. The application also integrates a quiz gamification approach through a specific section of the app, that allow the users to be more engaged, but also learn and deepen their knowledge about the topic. Finally, a notification system to prompt the user into inserting information, as well as a recommendation system (made through an LLM developed by other team members) provides personalized health and lifestyle recommendations. The thesis is then structured in 5 chapters: the first one deepen the health and well being topic, considering the guidelines that literature has found over the years of studying the topic. It then focuses on how technology can help us in achieving a better lifestyle, by considering which are the main software and hardware tools that can be used, such as smartphone application and wearable devices that allow to collect data and share them with application to create a more complete picture of the user's health. The second chapter covers the system, by firstly consider the updated requirements related to it (both functional and non-functional) and then moving to the system architecture, also redesigned consequently. The third chapter focuses on the technology stack, covering framework, programming languages, automation tools, the IDE, and backend development. The fourth chapter then talks about the overall implementation of the application, by focusing on my particular contribution. Finally, the fifth chapter talks about the performance analisys of the application, by considering different testing devices and different hardware specs to test the application performances, and then concludes the work by considering the results that have been achieved and the future work that can be done to improve the application.