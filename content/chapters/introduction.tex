\section*{Introduction}
\addcontentsline{toc}{section}{Introduction}
The work that is being presented in the next pages is based on two main arguments: the importance of following a healthy lifestyle and how technology can help in achieving this goal. Infact, our project work along with my colleagues concerned the implementation of an application suited for achieving a good lifestyle. The application, composed both by a frontend part and a backend one, allow the users to track their data regarding nutrition, as well as physical activity, sleep and stress data. \textcolor{red}{These data can be either inserted manually or collected through a wearable device, appropriately connected to the application. The application still requires the user to insert his basics information, such as username, age, weight, height. They are then all available in the profile settings, where the user can eventually change them}. The application also integrates a gamification approach through a specific section of the app, that allow the users to learn and deepen their knowledge about the topic, \textcolor{red}{granting a form of reward when users complete a specific task.} Finally, the application also involves the usage of recommendation as a way to provide suggestion about health and lifestyle to the user and improve his healthy journey. The thesys is then structured in 5 chapters: the first one deepen the health and well being topic, considering the guidelines that literature has found over the years of studying the topic. It then focuses on how technology can help us in achieving a better lifestyle, by considering which are the main software and hardware tools that can be used, such as smartphone application and wearable devices that allow to collect data and share with application to create a more complete picture of the user's health. The second chapter then moves into more tecnical aspects, by considering the technology stack that has been used to develop the application, starting from the used programming languages and automating tools, moving then to the IDE (Integrated Development Environment) and to the backend related aspect. The third chapter then talks about the overall implementation of the application, by focusing on my particular contribution. The fourth chapter then moves to the performance analisys of the application, by considering the different testing devices that have been used to test the application performances. Finally, the fifth chapter concludes the work by considering the results that have been achieved and the future work that can be done to improve the application.
\newpage