\section*{Introduction}
\addcontentsline{toc}{section}{Introduction}
The work that is being presented in the next pages is based on two main arguments: the importance of following a healthy lifestyle and how technology can help in achieving this goal. Infact, our project work along with my colleagues concerned the implementation of a mobile application suited for achieving a good lifestyle. The application, composed both by a frontend part and a backend one, allow the users to track their data regarding metrics like nutrition, as well as physical activity, sleep and emotional state. These data can be either inserted manually in an easy way or collected through a wearable device, and are intuitively displayed through charts. The application still requires the user to insert his basics information (username, age, sex, weight, height), which can be modified later in a dedicated section, together with his activity goals. The application also integrates a quiz gamification approach through a specific section of the app, that allow the users to be more engaged, but also learn and deepen their knowledge about the topic. Finally, a notification system to prompt the user into inserting information is included, as well as multi language support in order to provide a better user experience. In addition, a web application was also conceived to assist the admin into editing some application parameters and to help him into lesson and quizzes management. The thesis is then structured in 5 chapters: the first one deepen the health and well being topic, considering the guidelines that literature has found over the years of studying the topic. It then focuses on how technology can help us in achieving a better lifestyle, by considering which are the main software and hardware tools that can be used, such as smartphone application and wearable devices that allow to collect data and share them with application to create a more complete picture of the user's health. The second chapter covers the system, by firstly consider the requirements related to it (both functional and non-functional) and then moving to the system architecture, designed consequently. The third chapter focuses on the technology stack, covering framework, programming languages, automation tools, the IDE, and backend technlogies and integrations. The fourth chapter then talks about the practical implementation of the application system, by focusing on the backend system. Finally, the fifth chapter talks about the performance of the application system, and then concludes the work by considering the achieved results and the future work that can be done to improve the application.