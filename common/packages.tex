%%%%%%%%%%%%%%%%%%%%%%%%%%%%%%%%%%%%%%%%%%%%%%%%%%%%
\usepackage[english]{babel}
\usepackage[utf8]{inputenc}
\usepackage[T1]{fontenc}
\usepackage{lmodern}
% bibliography
\usepackage[hyperref=true,backref=true,backend=biber,maxbibnames=9,maxcitenames=2,style=ieee,citestyle=numeric,sorting=none]{biblatex}
% Customizing backref output
\DefineBibliographyStrings{english}{
    backrefpage = {\thefield{backref}}, % Formatting for single backref
    backrefpages = {\thefield{backref}}  % Formatting for multiple backrefs
}
% Adjusting the format of back-references
\renewbibmacro*{pageref}{%
    \iflistundef{pageref}
    {}
    {\printtext[brackets]{%
            \printlist{pageref}}}}
\renewbibmacro*{finentry}{%
    \iffieldundef{pageref}
    {}
    {\addspace\printtext[brackets]{\printlist{pageref}}}%
    \finentry}
% Redefine \cite to make text and number black
\usepackage{xcolor} % for color
\usepackage{etoolbox}
\usepackage{hyperref} % for hyperlinks
\makeatletter
\let\oldcite\cite
\AtBeginDocument{
    \renewcommand{\cite}[1]{\textcolor{black}{\hypersetup{citecolor=black}\oldcite{#1}}}
}
\makeatother

\usepackage{adjustbox} % to resize boxes by keeping the same aspect ratio
\usepackage{algorithm} % algorithm environment
\usepackage{algpseudocode} % improved pseudo-code
\usepackage{amsfonts}               %  AMS mathematical fonts
\usepackage{amsmath}
\usepackage{amssymb}                %  AMS mathematical symbols
\usepackage{bm}                     %  black/bold mathematical symbols
\usepackage{booktabs}               %  better tables
\usepackage[labelfont=bf]{caption} % font=footnotesize % to have reduced caption font size
\usepackage{csquotes}
\usepackage{enumitem} %left align the bulleted points
\usepackage{geometry} % FOR DEBUG USE [showframe]{geometry}
%\usepackage{glossaries} % to use acronyms and glossary, it has also glossaries-extra as extension, but commands are different
\usepackage[%
    toc, % puts the link in the ToC
    %record, % to use bib2gls
    abbreviations, % to load abbreviations / acronyms
    nonumberlist, % to avoid printing the numbers of the references in the acronyms page
]{glossaries-extra}
\usepackage{graphicx}               %  post-script images
\usepackage{wrapfig}
%\usepackage{iwona} % extra fonts, substitute standard ones
\usepackage{layout} % debug
\usepackage{listings} % to insert formatted code
\usepackage{lipsum} % for lorem ipsum text, not needed in the real work
\usepackage{makecell} % to change dimensions of cells, for math cases
\usepackage{mathtools} % for additional commands
\usepackage{mfirstuc} % to have capitalization capabilities
\usepackage[final]{microtype}      % microtypography, final lets latex use it also in bibliography
\usepackage{setspace} % for line spacing
\usepackage{multirow} % to allow for cells covering more than 1 row in tables
\usepackage{nicefrac}       % compact symbols for 1/2, etc.
%\usepackage[lofdepth,lotdepth]{subfig}
\usepackage{ragged2e} % for justifying text
\usepackage{rotating} % rotates things
\usepackage{siunitx} % support for SI units of measurement and number typesetting
\usepackage{subfig}
\usepackage{svg} % for svg support, works only if inkscape is installed, default for Overleaf v2
%\usepackage{subfigure}              %  subfigure compatibility, can be removed if subfig
\usepackage{tabularx} % equal-width columns in tables
\usepackage{textcomp} % extra fonts and symbols
\usepackage{url}            % simple URL typesetting
\usepackage{verbatim} % for extended verbatim support
\usepackage{xcolor} % to define colors and use standard CSS names add dvipsnames as option, but it clashes with xcolor loaded in toptesi, pay attention that if it goes in conflict with tikz/beamer, simply use \documentclass[usenames,dvipsnames]{beamer}, along with other custom options when defining the document class
\usepackage{xifthen}
\usepackage[nottoc]{tocbibind}
\usepackage{placeins}
\usepackage{hyperref} % must be loaded before glossaries-extra
\usepackage{cleveref}

% ------------------ \cref REDEFINITION with black color (only link now) -----------------------------
% ------------------ NORMAL -----------------------------
\crefformat{figure}{
    \unskip\textcolor{black}{ % fig. text color
        fig.~#2\textcolor{black}{#1}#3} % reference number color
    \unskip
}
\crefformat{table}{
    \unskip\textcolor{black}{ % table text color
        table~#2\textcolor{black}{#1}#3} % reference number color
    \unskip
}
\crefformat{section}{
    \unskip\textcolor{black}{ % section text color
        section~#2\textcolor{black}{#1}#3} % reference number color
    \unskip
}
\crefformat{equation}{
    \unskip\textcolor{black}{ % equation text color
        eq.~#2\textcolor{black}{#1}#3} % reference number color
    \unskip
}
% ------------------ CAPITALIZED -----------------------------
\Crefformat{figure}{
    \unskip\textcolor{black}{ % Fig. text color
        Fig.~#2\textcolor{black}{#1}#3} % reference number color
    \unskip
}
\Crefformat{table}{
    \unskip\textcolor{black}{ % Table text color
        Table~#2\textcolor{black}{#1}#3} % reference number color
    \unskip
}
\Crefformat{section}{
    \unskip\textcolor{black}{ % Section text color
        Section~#2\textcolor{black}{#1}#3} % reference number color
    \unskip
}
\Crefformat{equation}{
    \unskip\textcolor{black}{ % Eq. text color
        Eq.~#2\textcolor{black}{#1}#3} % reference number color
    \unskip
}